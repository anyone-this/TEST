% !TeX spellcheck = de_DE

%  ******************************************************************************
%  * @file      tex/Danksagung                                                  *
%  * @author    Mario Hesse                                                     *
%  * @version   v0.1.0                                                          *
%  * @date      31.01.2019                                                      *
%  ******************************************************************************

\section*{Danksagung}

Es gehört zum guten Ton, sich bei der Fertigstellung einer wissenschaftlichen Arbeit bei wichtigen Menschen zu bedanken. Diese Menschen können beispielsweise deine Eltern, dein*e Partner*in und dein*e Betreuer*in sein. Es gibt allerdings auch Wissenschaftler*innen, die sich bei ihrer Katze oder ihrem Goldfisch bedanken oder simpel \textit{Danke an alle.} schreiben. In diesem Teil der Arbeit kannst du zwar wirklich alles schreiben, was du willst, allerdings ist das auch der Teil an dem andere Leute, sehen können wie du tickst. Daher rate ich dazu dieses Kapitel nicht all zu extravagant auszustaffieren.
\vspace{3em}

Göttingen, den \today
