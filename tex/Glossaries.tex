% !TeX spellcheck = de_DE

%  ***********************************************
%  * Dokument: tex/Glossaries           *
%  * Editor:   Mario Hesse                       *
%  * Version:  v0.1.0                           *
%  * Date:     31.01..2019                        *
%  ***********************************************

% Symbole
% -------
\newglossaryentry{symb:U}{
	name=$U$,
	description={Spannung},
	sort=symbolU,
	type=symbolslist
}
\newglossaryentry{symb:I}{
	name=$I$,
	description={Strom},
	sort=symbolI,
	type=symbolslist
}
\newglossaryentry{symb:R}{
	name={$R$},
	description={Widerstand},
	sort=symbolR,
	type=symbolslist
}
\newglossaryentry{symb:C}{
	name={$C$},
	description={Kapazität},
	sort=symbolC,
	type=symbolslist
}
\newglossaryentry{symb:f}{
	name={$f_g$},
	description={Frequenz},
	sort=symbolf,
	type=symbolslist
}
\newglossaryentry{symb:pi}{
	name={$\pi$},
	description={Kreiszahl ($\approx$3,142)},
	sort=symbolpi,
	type=symbolslist
}


 
% Abkürzungen
% -----------
\newacronym
	{ems}{EMS}{elektromagnetische Störung}
\newacronym
	{emv}{EMV}{elektromagnetische Verträglichkeit}
\newacronym
	{sps}{SPS}{\gls{SpeicherprogrammierbareSteuerung}}
\newacronym
	{hawk}{HAWK}{Hochschule für angewandte Wissenschaft und Kunst Hildesheim/Holzminden/Göttingen}
\newacronym[description={Analog-Digital-Converter (Analog-Digital-Wandler)}]
	{adc}{ADC}{Analog-Digital-Converter}


 
% Glossar
% -------
\newglossaryentry{glos:dbd}{
	name={Dielektrisch behinderte Entladung},
	description={ist eine besondere Form der \gls{Plasma} Entladung, bei der ein kaltes Plasma mit sehr niedriger Energie auch bei Atmosphärendruck gezündet werden kann. Sie zeichnet sich durch ein Dielektrikum in der Hochspannungsmasche aus.}
}
\newglossaryentry{Steuerung}{
	name={Steuerung},
	description={beschreibt einen Prozess, der eine Größe (z.B. eine Plasmaentladung) ausgibt. Die Größe wird mit festen Werten von einem Anwender eingestellt.}
}
\newglossaryentry{Regelung}{
	name={Regelung},
	description={beschreibt einen Prozess, der eine Größe (z.B. eine Plasmaentladung) ausgibt. Der Effekt, den diese Größe hervorruft, wird gemessen und in den Prozess zurückgekoppelt, um damit die Größe so zu verändern, dass der gemessene Effekt dem Anwenderwunsch entspricht und konstant gehalten wird.}
	}
\newglossaryentry{Mikrocontroller}{
	name={Mikrocontroller},
	description={ist ein integrierter Schaltkreis, der alle elementaren Funktionseinheiten eines Computersystems auf einem Chip vereint. Heutzutage bildet er in den meisten handelsüblichen elektronischen Geräten die intelligente Steuereinheit.}
}
\newglossaryentry{SpeicherprogrammierbareSteuerung}{
	name={Speicherprogrammierbare Steuerung},
	description={ist eine Art Industriecomputer, der dazu geschaffen ist große, mittlere und kleine Industrieanlagen zu steuern.},
	plural={Speicherprogramierbare Steuerungen}
}
\newglossaryentry{Plasma}{
	name={Plasma},
	description={ist der 4. Aggregatzustand. Er entsteht, wenn einem Gas zusätzlich Energie zugeführt wird. In diesem Zustand wandeln sich Atome und Moleküle ständig in Ionen und Elektronen um und umgekehrt.}
}






